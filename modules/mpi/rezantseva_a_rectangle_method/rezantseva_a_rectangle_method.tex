\documentclass[12pt]{article}
\usepackage[T2A]{fontenc}
\usepackage[utf8]{inputenc}
\usepackage[russian]{babel}
\usepackage{hyperref}
\usepackage{geometry}
\usepackage{amsmath}
\usepackage{amsfonts}
\usepackage{graphicx}

\usepackage{xcolor}
\geometry{top=2cm,bottom=2cm,left=2cm,right=2cm}
\linespread{1.5}

%C++ code 
\usepackage{listings}
\usepackage{color}
% параметры стиля
\lstdefinestyle{mystyle}{
	basicstyle=\footnotesize,
	breakatwhitespace=false, 
	breaklines=true,
	captionpos=b,
	keepspaces=true, 
	numbers=left,
	numbersep=5pt,  % номера строк
	showspaces=false, % не показывать пробелы
	showstringspaces=false,
	showtabs=false,                  
	tabsize=2
}

% применяем стиль
\lstset{style=mystyle}
% используем заданный нами моноширинный шрифт
\lstset{basicstyle=\footnotesize\ttfamily,breaklines=true}




\begin{document}
\begin{titlepage}
	\begin{center}
		\large
		{МИНИСТЕРСТВО НАУКИ И ВЫСШЕГО ОБРАЗОВАНИЯ\\ РОССИЙСКОЙ ФЕДЕРАЦИИ}
		
		Федеральное государственное автономное образовательное учреждение высшего образования
		\vspace{0.5cm}
		
		\textbf{<<Национальный исследовательский Нижегородский государственный университет им. Н.И. Лобачевского>>}\\
		(ННГУ)\\
		\vspace{1cm}
		\textbf{Институт информационных технологий, математики и механики}\\
		\vspace{1cm}
		
		
		\vfill
		
		\vfill
		
		\Large
		\textbf{Отчёт по лабораторной работе на тему} \\
		\textbf{<<Вычисление многомерных интегралов с использованием многошаговой схемы (метод прямоугольников).>>}
		{\LARGE 
		}
		\bigskip
		
		
	\end{center}
	\vfill
	
	\hfill\begin{minipage}{0.4\textwidth}
		\textbf{Выполнил:} 

            студентка группы 3822Б1ФИ1

            \underline{\hspace{3cm}} Резанцева А. А. \bigskip
            
	\end{minipage}%
	
	\hfill\begin{minipage}{0.4\textwidth}
		\textbf{Проверил:}

           доцент кафедры ВВСП, к.т.н.
           
           \underline{\hspace{3cm}} Сысоев А.В.\bigskip
	\end{minipage}%
	\vfill
	
	\begin{center}
		Нижний Новгород\\
		2024 г.
	\end{center}
\end{titlepage}
\thispagestyle{empty} % убираем колонтитулы с этой страницы
\tableofcontents

\newpage

\section*{Введение}
\addcontentsline{toc}{section}{Введение}
    Вычисление многомерных интегралов является важной задачей в различных областях науки и техники, включая физику, инженерию и т.д. Многомерные интегралы позволяют моделировать сложные системы и процессы, где необходимо учитывать взаимодействие нескольких переменных. Однако аналитическое решение таких интегралов часто оказывается крайне трудоемким или невозможным. В таких случаях на помощь приходят численные методы, которые позволяют получить приближенные значения интегралов с заданной точностью.
	
\vspace{1 cm}

    Одним из простейших и наиболее распространенных численных методов для вычисления многомерных интегралов является метод прямоугольников, также известный как метод Римана. Этот метод основывается на разбиении области интегрирования на небольшие прямоугольники и оценке значения интеграла как суммы значений функции в этих точках, умноженных на объем соответствующих прямоугольников.

\vspace{1 cm}

    Метод прямоугольников обладает рядом преимуществ, включая простоту реализации и возможность применения к интегралам с произвольной сложностью. Однако его точность зависит от размера разбиения: чем меньше прямоугольники, тем более точным будет результат. В то же время, уменьшение размера прямоугольников приводит к увеличению вычислительных затрат, что требует оптимального выбора параметров разбиения.

\vspace{1 cm}
    Целью данной лабораторной работы является исследование метода прямоугольников для вычисления многомерных интегралов, а также разработка и реализация этого алгоритма в последовательной и параллельной версиях. В рамках работы будет проведено сравнение их эффективности.

\newpage

\section{Постановка задачи}
Дано многомерное пространство $\mathbb{R}^n$ и функция $f: \mathbb{R}^n \to \mathbb{R}$, которую необходимо интегрировать по некоторому многомерному объему $V$. Объем $V$ определяется как прямоугольник, заданный границами по каждой из $n$ координат.

Формально, задача может быть описана следующим образом:

\vspace{0.5 cm}

\textbf{Выходные данные}
\begin{itemize}
    \item $n$ — размерность пространства.
    \item $f(x_1, x_2, \ldots, x_n)$ — функция, которую требуется интегрировать.
    \item $V$ — многомерный прямоугольник, заданный границами:
    \[
    V = \{(x_1, x_2, \ldots, x_n) \in \mathbb{R}^n \mid a_i \leq x_i \leq b_i, \, i = 1, 2, \ldots, n\}
    \]
    где $a_i$ и $b_i$ — нижние и верхние границы интегрирования по координате $x_i$.
\end{itemize}

\vspace{0.5 cm}

\textbf{Выходные данные} \\
    Приближенное значение многомерного интеграла:
\[
I \approx \int_V f(x_1, x_2, \ldots, x_n) \, dV
\]

Для вычисления многомерного интеграла с использованием метода прямоугольников, объем $V$ разбивается на $N$ подпрямоугольников. Каждый подпрямоугольник имеет размеры $\Delta x_1, \Delta x_2, \ldots, \Delta x_n$, где:
\[
\Delta x_i = \frac{b_i - a_i}{N_i}, \quad i = 1, 2, \ldots, n
\]
где $N_i$ — количество разбиений по координате $x_i$.

\newpage

\section{Описание алгоритма}
Алгоритм вычисления многомерных интегралов с использованием многошаговой схемы  заключается в делении многомерного пространства на подпрямоугольники и оценке интеграла путем суммирования значений функции в определенных точках. Каждый подпрямоугольник имеет объем, который учитывается при суммировании. Это позволяет значительно упростить вычисления, особенно для функций, которые сложно интегрировать аналитически.\\

\subsection*{Алгоритм}
\begin{enumerate}
    \item  Определить количество разбиений $N$ по каждой координате.
    \item  Вычислить размеры подпрямоугольников:
    \[
    \Delta x_i = \frac{b_i - a_i}{N_i}, \quad i = 1, 2, \ldots, n
    \]
    \item  Для каждого подпрямоугольника определить его центр (или другой подходящий выбор точки внутри):
    \[
    c_{i} = a_i + \left(j_i + 0.5\right) \Delta x_i, \quad j_i = 0, 1, \ldots, N_i - 1
    \]
    \item  Вычислить значение функции $f$ в выбранной точке для каждого подпрямоугольника:
    \[
    f(c_1, c_2, \ldots, c_n)
    \]
    \item  Вычислить объем каждого подпрямоугольника:
    \[
    V_j = \prod_{i=1}^{n} \Delta x_i
    \]
    \item Суммировать все значения функции, умноженные на объем подпрямоугольников:
    \[
    I \approx \sum_{j=1}^{N} f(c_1, c_2, \ldots, c_n) \cdot V_j
    \]
\end{enumerate}

\subsection*{Сложность алгоритма}
Сложность алгоритма зависит от числа разбиений $N_1$ ... $N_i$ и размерности $n$ пространства. В общем случае, если мы имеем $N$ разбиений по каждой из $n$ координат, то общее количество подпрямоугольников будет равно $N^n$. Таким образом, сложность алгоритма можно оценить как $O\left(\prod\limits_{i=1}^{n} N_i\right)$, что делает его неэффективным для больших значений $N$ и $n$. Однако в практических задачах часто применяются адаптивные методы и оптимизации, которые позволяют снизить вычислительные затраты, сохраняя при этом приемлемую точность.

\section{Описание схемы распараллеливания}
Образуем пространство точек (прямоугольников) с размером \( \prod\limits_{i=1}^{n} N_i \) и распределяем пространство точек(прямоугольников)  между процессами. Для каждого прямоугольника мы должны получить точку ($x_1,...,x_n$) - набор параметров.Эти параметры мы получаем из адреса (id)  в глобальном пространстве точек по принципу системы координат составленной из размеров разбиений. 

После разбиения точек пространства по процессам и получения значения этих точек можно приступить к вычислению значения функции в этих точках. Параллелизм заключается в том, что каждый процесс вычисляет функцию только для своего набора точек. Значения функций, подсчитаные для каждой точки, суммируются на своих процессах, затем  отправляются на корневой, суммируясь с локальными суммами других наборов точек, вычисленных на остальных процессах.  В итоге формируется общая сумма значений функций во всех точках пространства. Это работает,  так как верно тождество: 
    \[
     \sum_{j=1}^{N}( f(c_1, c_2, \ldots, c_n) \cdot \prod_{i=1}^{n} \Delta x_i) = \prod_{i=1}^{n} \Delta x_i\sum_{j=1}^{N} f(c_1, c_2, \ldots, c_n)
    \]

Домнажение на все $\Delta x_i$ происходит только на корневом процесске, после вычисляения итоговой суммы функций.

\section{Описание MPI-версии программной реализации}
\subsection*{Подготовка и распределение данных}
В методе \textbf{pre-processing()} на корневом процессе заполняются поля данными о количестве разбиений, пределами интегрирования, а также вычисляется размер пространства прямоугольников. Далее эти данные рассылаются с помощью \textbf{boost::mpi::broadcast()}, таким образом, каждый прооцесс знает пределы интегрирования и соответсвующие разбиения.

\subsection*{Локальные вычисления}
Каждый процесс вычисляет какой набор точек он должен обработать в зависимости от номера процесса и размера пространства прямоугольника. Дальше он вычисляет функцию от каждой точки  и суммирует с функциями от других точек из набора. 

\subsection*{Сбор результатов с процессов}
Далее, с помощью \textbf{  boost::mpi::reduce(<коммуникатор>, <локальная сумма>, <результирующая сумма>, std::plus<>(), 0)}  мы собиарем на корневом процессе с рангом 0 все локальные суммы других процессов, суммируя их.

\subsection*{Вычисление итогово результата}
Получив итоговую сумму функций, на корневом процессе домнажаем ее на длины отрезков наших пределов интегрирования для получения итогово результата. 
\newpage

\section{Эксперименты}
Тестовая задача заключалась в вычислении трёхмерного интеграла для функции:\\
\begin{center}
$f(x,y,z) = x^2-2y+8z$ ,
\end{center}
 где $x \in [-6, 10]$, $y \in [5, 27]$, $z \in [15, 32]$. Точность интегрирования: $0.0001$.

\subsection*{Результаты экспериментов}
\begin{tabular}{|c|c|c|c|}
    \hline
    Число процессов & Размер разбиения      & Время выполнения (с) \\ \hline
    1               & 100                   & 0,048                   \\ \hline
    2               & 100                   & 0.032                  \\ \hline
    3               & 100                   & 0.024                 \\ \hline
    5               & 100                   & 0.019                   \\ \hline
    16              & 100                   & 0.017                  \\ \hline
    1               & 250                   & 0.631                    \\ \hline
    2               & 250                   & 0.378                   \\ \hline
    3               & 250                   & 0.305                  \\ \hline
    5               & 250                   & 0.251                    \\ \hline
    16              & 250                   & 0.209                    \\ \hline
    1               & 250                   & 4.586                    \\ \hline
    2               & 500                   & 2.882                    \\ \hline
    3               & 500                   & 2.249                    \\ \hline
    5               & 500                   & 1.950                   \\ \hline
    16              & 500                   & 1.556                    \\ \hline
\end{tabular}

\subsection*{Анализ результатов}
Время выполнения теста уменьшается с увеличением числа процессов, что подтверждает корректность параллельной реализации. Также при большем разбиении и при увеличении числа процессов ускорение
больше.

\newpage
\section*{Заключение}
\addcontentsline{toc}{section}{Заключение}
В данной работе был рассмотрен и реализован  метод вычисления многомерных интегралов с использованием многошаговой схемой методом прямоугольников. Реализация включает в себя  последовательный и параллельный варианты. Были проведены эксперименты, в ходе которых было выявлено, что использование параллельного программирование помогает ускорить выполнение задачи.

\newpage
\section*{Список литературы}
\addcontentsline{toc}{section}{Список литературы}
\begin{enumerate}
    \item \url{https://neerc.ifmo.ru/wiki/index.php?title=%D0%9E%D0%BF%D1%80%D0%B5%D0%B4%D0%B5%D0%BB%D0%B5%D0%BD%D0%B8%D0%B5_%D0%B8%D0%BD%D1%82%D0%B5%D0%B3%D1%80%D0%B0%D0%BB%D0%B0_%D0%A0%D0%B8%D0%BC%D0%B0%D0%BD%D0%B0,_%D0%BF%D1%80%D0%BE%D1%81%D1%82%D0%B5%D0%B9%D1%88%D0%B8%D0%B5_%D1%81%D0%B2%D0%BE%D0%B9%D1%81%D1%82%D0%B2%D0%B0}{ Интеграл Римена}.
    \item \url{https://neerc.ifmo.ru/wiki/index.php?title=%D0%9E_%D0%BC%D0%BD%D0%BE%D0%B3%D0%BE%D0%BA%D1%80%D0%B0%D1%82%D0%BD%D1%8B%D1%85_%D0%B8%D0%BD%D1%82%D0%B5%D0%B3%D1%80%D0%B0%D0%BB%D0%B0%D1%85}{ Многомерные интегралы}.
    \item \url{https://www.youtube.com/watch?v=JXh9AQkKmsw}{ Вычисление интегралов с помощью метода Римана}.
\end{enumerate}
\newpage
\section*{Приложение}
\addcontentsline{toc}{section}{Приложение}
\subsection*{opsMpiRezA.hpp}
\addcontentsline{toc}{subsection}{opsMpiRezA.hpp}
\begin{lstlisting}[language=C++]
#pragma once
#include <gtest/gtest.h>

#include <boost/mpi/collectives.hpp>
#include <boost/mpi/communicator.hpp>
#include <boost/serialization/utility.hpp>
#include <boost/serialization/vector.hpp>
#include <cmath>
#include <functional>
#include <vector>

#include "core/task/include/task.hpp"

#define func double(const std::vector<double> &)

namespace rezantseva_a_rectangle_method_mpi {

bool check_integration_bounds(std::vector<std::pair<double, double>> *ib);

template <class Archive, typename T1, typename T2>
void serialize(Archive &ar, std::pair<T1, T2> &p, const unsigned int version) {
  ar & p.first;
  ar & p.second;
}

class RectangleMethodSequential : public ppc::core::Task {
 public:
  explicit RectangleMethodSequential(std::shared_ptr<ppc::core::TaskData> taskData_, std::function<func> f)
      : Task(std::move(taskData_)), func_(std::move(f)) {}

  bool pre_processing() override;
  bool validation() override;
  bool run() override;
  bool post_processing() override;

 private:
  double result_{};
  std::vector<std::pair<double, double>> integration_bounds_;
  std::vector<int> distribution_;
  std::function<func> func_;
};

class RectangleMethodMPI : public ppc::core::Task {
 public:
  explicit RectangleMethodMPI(std::shared_ptr<ppc::core::TaskData> taskData_, std::function<func> f)
      : Task(std::move(taskData_)), func_(std::move(f)) {}

  bool pre_processing() override;
  bool validation() override;
  bool run() override;
  bool post_processing() override;

 private:
  boost::mpi::communicator world;
  double result_{};
  int num_processes_ = 0;
  std::vector<std::pair<double, double>> integration_bounds_;
  std::vector<int> distribution_;
  std::function<func> func_;
  int n_;
  std::vector<double> widths_;
  int total_points_{};
};
}  // namespace rezantseva_a_rectangle_method_mpi

\end{lstlisting}
\subsection*{opsMpiRezA.cpp}
\addcontentsline{toc}{subsection}{opsMpiRezA.cpp}
\begin{lstlisting}[language=C++]
// mpi cpp rectangle method
#include "mpi/rezantseva_a_rectangle_method/include/ops_mpi_rez_a.hpp"

bool rezantseva_a_rectangle_method_mpi::check_integration_bounds(std::vector<std::pair<double, double>>* ib) {
  if (ib == nullptr) {
    std::cerr << "Error: bounds pointer is null." << std::endl;
    return false;
  }

  bool result = std::ranges::all_of(*ib, [](const auto& bound) { return bound.first < bound.second; });

  return result;
}

bool rezantseva_a_rectangle_method_mpi::RectangleMethodSequential::validation() {
  internal_order_test();
  auto* bounds = reinterpret_cast<std::vector<std::pair<double, double>>*>(taskData->inputs[0]);
  return (taskData->inputs.size() == 2 && taskData->outputs_count[0] == 1 &&
          (taskData->inputs_count[0] == taskData->inputs_count[1]) && check_integration_bounds(bounds));
}

bool rezantseva_a_rectangle_method_mpi::RectangleMethodSequential::pre_processing() {
  internal_order_test();

  auto* bounds = reinterpret_cast<std::vector<std::pair<double, double>>*>(taskData->inputs[0]);
  integration_bounds_.assign(bounds->begin(), bounds->end());

  auto* distribution_ptr = reinterpret_cast<std::vector<int>*>(taskData->inputs[1]);
  distribution_.assign(distribution_ptr->begin(), distribution_ptr->end());

  result_ = 0.0;

  return true;
}

bool rezantseva_a_rectangle_method_mpi::RectangleMethodSequential::run() {
  internal_order_test();
  int dimension = distribution_.size();
  std::vector<double> widths(dimension);
  int64_t rectangles = 1;

  for (int i = 0; i < dimension; i++) {
    widths[i] = (integration_bounds_[i].second - integration_bounds_[i].first) / static_cast<double>(distribution_[i]);
    rectangles *= distribution_[i];
  }
  std::vector<double> params(dimension);

  for (int i = 0; i < rectangles; i++) {
    int x = i;
    for (int j = 0; j < dimension; j++) {
      int idDimention = x % distribution_[j];
      params[j] = integration_bounds_[j].first + idDimention * widths[j] + widths[j] / 2;
      x /= distribution_[j];
    }
    result_ += func_(params);
  }
  for (int i = 0; i < dimension; i++) {
    result_ *= widths[i];
  }
  return true;
}

bool rezantseva_a_rectangle_method_mpi::RectangleMethodSequential::post_processing() {
  internal_order_test();
  reinterpret_cast<double*>(taskData->outputs[0])[0] = result_;
  return true;
}

bool rezantseva_a_rectangle_method_mpi::RectangleMethodMPI::validation() {
  internal_order_test();
  bool flag = true;

  if (world.rank() == 0) {
    auto* bounds = reinterpret_cast<std::vector<std::pair<double, double>>*>(taskData->inputs[0]);
    flag = (taskData->inputs.size() == 2 && taskData->outputs_count[0] == 1 &&
            (taskData->inputs_count[0] == taskData->inputs_count[1]) && check_integration_bounds(bounds));
  }
  return flag;
}

bool rezantseva_a_rectangle_method_mpi::RectangleMethodMPI::pre_processing() {
  internal_order_test();

  if (world.rank() == 0) {
    auto* bounds = reinterpret_cast<std::vector<std::pair<double, double>>*>(taskData->inputs[0]);
    integration_bounds_.assign(bounds->begin(), bounds->end());

    auto* distribution_ptr = reinterpret_cast<std::vector<int>*>(taskData->inputs[1]);
    distribution_.assign(distribution_ptr->begin(), distribution_ptr->end());

    num_processes_ = world.size();
    n_ = integration_bounds_.size();
    total_points_ = 1;
    widths_.assign(distribution_.size(), 0.0);
    for (int i = 0; i < n_; i++) {
      widths_[i] =
          (integration_bounds_[i].second - integration_bounds_[i].first) / static_cast<double>(distribution_[i]);
      if (i != n_ - 1) {
        total_points_ *= distribution_[i];
      }
    }
  }

  boost::mpi::broadcast(world, n_, 0);
  boost::mpi::broadcast(world, total_points_, 0);
  boost::mpi::broadcast(world, num_processes_, 0);
  boost::mpi::broadcast(world, distribution_, 0);
  boost::mpi::broadcast(world, widths_, 0);
  boost::mpi::broadcast(world, integration_bounds_, 0);

  result_ = 0.0;

  return true;
}

bool rezantseva_a_rectangle_method_mpi::RectangleMethodMPI::run() {
  internal_order_test();

  int remainder = total_points_ % num_processes_;
  int delta = total_points_ / num_processes_ + (world.rank() < remainder ? 1 : 0);

  std::vector<std::vector<double>> params(delta);

  int offset = 0;
  if (world.rank() < remainder) {
    offset = world.rank() * delta;
  } else {
    offset = remainder * (delta + 1) + (world.rank() - remainder) * delta;
  }

  for (int i = 0; i < delta; i++) {
    int x = offset + i;
    for (int j = 0; j < n_ - 1; j++) {
      int idDimention = x % distribution_[j];
      params[i].push_back(integration_bounds_[j].first + idDimention * widths_[j] + widths_[j] / 2);
      x /= distribution_[j];
    }
  }

  double local_sum = 0.0;
  for (int i = 0; i < delta; i++) {
    for (int j = 0; j < distribution_[n_ - 1]; j++) {
      params[i].push_back(integration_bounds_[n_ - 1].first + j * widths_[n_ - 1] + widths_[n_ - 1] / 2);
      local_sum += func_(params[i]);
      params[i].pop_back();
    }
  }

  boost::mpi::reduce(world, local_sum, result_, std::plus<>(), 0);
  for (int i = 0; i < n_; i++) {
    result_ *= widths_[i];
  }

  return true;
}

bool rezantseva_a_rectangle_method_mpi::RectangleMethodMPI::post_processing() {
  internal_order_test();
  if (world.rank() == 0) {
    reinterpret_cast<double*>(taskData->outputs[0])[0] = result_;
  }
  return true;
}
\end{lstlisting}

\end{document}